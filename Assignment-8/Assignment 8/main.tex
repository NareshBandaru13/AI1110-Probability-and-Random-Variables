\documentclass{beamer}
\usetheme{CambridgeUS}

\setbeamertemplate{caption}[numbered]{}

\usepackage{enumitem}
\usepackage{amsmath}
\usepackage{amssymb}
\usepackage{gensymb}
\usepackage{graphicx}
\usepackage{txfonts}

\def\inputGnumericTable{}

\usepackage[latin1]{inputenc}                                 
\usepackage{color}                                            
\usepackage{array}                                            
\usepackage{longtable}                                        
\usepackage{calc}                                             
\usepackage{multirow}                                         
\usepackage{hhline}                                           
\usepackage{ifthen}
\usepackage{caption}

\title{AI1110 \\ Assignment 8}
\author{Bandaru Naresh Kumar \\ AI21BTECH11006}
\date{}
\begin{document}
	% The title page
	\begin{frame}
		\titlepage
	\end{frame}
	
	% The table of contents
	\begin{frame}{Outline}
    		\tableofcontents
	\end{frame}
	
	% The question
	\section{Question}
	\begin{frame}{Exercise 9.34}
       Show that the power spectrum of an SSS process X(t) equals\\
       $ S(\omega) = \int_{-\infty}^{\infty}\int_{-\infty}^{\infty}x_1x_2G(x_1,x_2;\omega)\,dx_1\,dx_2 $ 
	\end{frame}
	
	% The solution
	\section{Solution}
	\begin{frame}{Solution}
	    We have,\\
	    \begin{align}
	    G(x_1,x_2;\omega) = \int_{-\infty}^{\infty}f(x_1,x_2;\tau)e^{-j\omega\tau}\,d\tau
	    \end{align}
	    Also,
	  \begin{align}
	  R(\tau) &= E\{X(t+\tau)X(t)\}\\
	          &= \int_{-\infty}^{\infty}\int_{-\infty}^{\infty}x_1x_2f(x_1,x_2;\tau)\,dx_1\,dx_2
	  \end{align}
	 \end{frame}
	 
	 % The final answer
	\section{Answer}
	\begin{frame}{Answer}
	   Hence,\\
	   \begin{align}
	   S(\omega) &= \int_{-\infty}^{\infty}R(\tau)e^{-j\omega\tau}\,d\tau\\
	             &= \int_{-\infty}^{\infty}e^{-j\omega\tau}\int_{-\infty}^{\infty}\int_{-\infty}^{\infty}x_1x_2f(x_1,x_2;\tau)\,dx_1\,dx_2\,d\tau\\
	             &= \int_{-\infty}^{\infty}\int_{-\infty}^{\infty}x_1x_2\int_{-\infty}^{\infty}e^{-j\omega\tau}f(x_1,x_2;\tau)\,d\tau\,dx_1\,dx_2\\
	             &= \int_{-\infty}^{\infty}\int_{-\infty}^{\infty}x_1x_2G(x_1,x_2;\omega)\,dx_1\,dx_2
	   \end{align}
	   
	\end{frame}
	
\end{document}