\documentclass{beamer}
\usetheme{CambridgeUS}

\setbeamertemplate{caption}[numbered]{}

\usepackage{enumitem}
\usepackage{amsmath}
\usepackage{amssymb}
\usepackage{gensymb}
\usepackage{graphicx}
\usepackage{txfonts}

\def\inputGnumericTable{}

\usepackage[latin1]{inputenc}                                 
\usepackage{color}                                            
\usepackage{array}                                            
\usepackage{longtable}                                        
\usepackage{calc}                                             
\usepackage{multirow}                                         
\usepackage{hhline}                                           
\usepackage{ifthen}
\usepackage{caption}

\title{AI1110 \\ Assignment 10}
\author{Bandaru Naresh Kumar \\ AI21BTECH11006}
\date{}
\begin{document}
	% The title page
	\begin{frame}
		\titlepage
	\end{frame}
	
	% The table of contents
	\begin{frame}{Outline}
    		\tableofcontents
	\end{frame}
	
	% The question
	\section{Question}
	\begin{frame}{Exercise 12.2}
       Show that if a process is normal and distribution-ergodic,then it is also mean-ergodic.
	\end{frame}
	
	% The solution
	\section{Solution}
	\begin{frame}{Solution}
	    The process X(t) is normal and such that:\\
	    \begin{align}
	    F(x,x;\tau) \rightarrow F^2(x) \text{as} \tau\rightarrow\infty
	    \end{align}
	    We shall show that it is mean-ergodic.It suffices to show         that: \\
       $C(\tau)\rightarrow0$  as $\tau\rightarrow\infty$        	     
	 \end{frame}
	  
	 %The prooof
	 \section{Proof}
	 \begin{frame}{Proof}
	 We can assume that $\eta=0$ and $C(0)=1$.\\
	 With this assumption,\\
	    The RVs $X(t+\tau)$ and X(t) are N(0,0;1,1;r) where $r=r(\tau)=C(\tau)$ is the autocovariance of X(t).\\
	 Hence,\\
	 \begin{align}
	 f(x_1,x_2;\tau) &= \dfrac{1}{2\Pi\sqrt{1-r^2}}exp\left\lbrace-\dfrac{1}{2(1-r^2)}(x_1^2-2rx_1x_2+x_2^2) \right\rbrace\\ 
	                 &= \dfrac{1}{2\Pi\sqrt{1-r^2}}exp\left\lbrace-\dfrac{1}{2(1-r^2)}(x_1-rx_2)^2 \right\rbrace e^{-\dfrac{x_2^2}{2}}
     \end{align}	    
     \end{frame}	  
	 
	 \begin{frame}
	 Clearly,$f(x,y) = f(y,x)$\\
	 Hence,\\
	 \begin{align}
	 F(x+dx,x+dx;\tau)-F(x,x;\tau) = 2\int_{-\infty}^{x}f(\xi,x)\,d\xi dx\\
	                               = \dfrac{1}{\Pi\sqrt{1-r^2}}\int_{-\infty}^{x}exp\left\lbrace-\dfrac{1}{2(1-r^2)}(\xi-xr)^2 \right\rbrace\,d\xi e^{-\dfrac{x_2^2}{2}}dx
	 \end{align}
	 \end{frame}
	 
	 % The final answer
	\section{Answer}
	\begin{frame}{Answer}
	   Furthermore,
	   \begin{align}
	   F^2(x+dx)-F^2(x) = 2F(x)f(x)dx
	   \end{align}
	  From (1) and (6);\\
	  $G\left( \dfrac{x-rx}{\sqrt{1-r^2}}\right)  \rightarrow G(x)$ as $\tau \rightarrow \infty$\\
	  Hence,\\
	  $r(\tau) \rightarrow 0$ as $\tau \rightarrow \infty$	  
	\end{frame}
\end{document}